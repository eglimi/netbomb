\chapter{Software Architektur}

\section{Logische Sicht}
Das System besteht aus prim"ar 4 Schichten (3  Schichten - Architektur plus Netzwerk). Diese beinhalten:
\begin{description}
\item[Schicht 1] Graphisches User Interface (GUI)
\item[Schicht 2] Problem Domain (PD)
\item[Schicht 3] Datenhaltung (DH)
\item[Schicht 4] Netzwerk (bestehend aus Client und Server)
\end{description}

Diese Architektur haben wir gew"ahlt um eine klare Trennung zwischen den einzelnen Programmteilen zu erreichen. \\
Das graphische User Interface ist die Schnittstelle zum Benutzer. "Uber dieses kann er Eingaben machen oder in unserem Fall erfolgt die Steuerung
der Spielfigur "uber das GUI. \\
Die PD ist die Schnittstelle zwischen GUI und DH. Das heisst sie bekommt und verarbeitet Befehle vom GUI, schreibt Daten in die DH und
ruft Funktionen in der Netzwerkschicht auf. \\
Die Datenhaltung ist daf"ur zust"andig, Daten, die konsistent sein m"ussen zu speichern, damit sie bei einem Neustarten des Spiels wieder
zur Verf"ugung stehen. \\
Die Netzwerkschicht regelt die Daten"ubertragung zwischen dem Server und den Clients.

\section{Prozess}


\subsection{Netzwerk}
Das Netzwerk besteht aus einem Server und bis zu vier Clients die miteinander kommunizieren. Dabei kann jeder Client auch Server sein, das heisst
zu Beginn des Spiels entscheidet der Spieler, ob er Server und Client oder nur Client ist. Es kann nur ein Spieler Server sein.
Ist ein Spieler Server und Client, werden die Daten der Mitspieler zu ihm "ubermittelt. Das geschieht folgendermassen: \\
Alle 20\(ms\) wird die aktuelle Bewegung eines Clients, also eines Spielers zum Server "ubermittelt. Diese Daten werden vom Server
entgegengenommen und \textit{allen} Clients "ubermittelt. Das heisst auch dem Client, der als Server fungiert. Damit k"onnen alle
Clients die Bewegung der Mitspieler berechnen und haben somit die aktuelle Position. Zus"atzlich zur Bewegungsrichtung wird noch
"ubermittelt, ob ein Spieler eine Bombe gelegt hat und wenn ja, wo sich diese befindet. Zus"atzlich zu dieser "Ubertragung wird alle 100\(ms\) die
aktuelle Position des Clients zum Server "ubermittelt. Das heisst nicht nur die Bewegungsrichtung, sondern die tats"achliche Position.
Diese Daten werden ebenfalls von jedem Client an den Server "ubermittelt und gleich wie vorher schickt der Server diese Daten an alle
Clients zur"uck. Damit verhindern wir allf"allige Ungenauigkeiten bzw. wir k"onnen damit den Fehler beheben, der durch zeitliche Verz"ogerungen
entsteht. Damit wir die zeitliche Verz"ogerung an sich in den Griff kriegen, senden wir alle 200\(ms\) einen Ping an jeden Client. Mit der Zeit,
die uns dieser Ping liefert, kann der Server feststellen, um wieviel sich die Position des Clients seit dem versenden seiner Position ge"andert hat.
\\ Mit der Tatsache, dass kein Client seine Spielposition selber berechnet erreichen wir faire Bedingungen falls die "Ubertragungsgeschwindigkeit
zwischen den einzelnen Clients und dem Server unterschiedlich sein sollte.


