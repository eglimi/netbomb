\chapter{Supplementary Specification}

\section{Functionality}

\subsection{Error handling}
Fehler werden dem Spieler mittels einer Fehlermeldung gemeldet.
Die Fehler werden nicht persistent in einer log-datei gespeichert.

\subsection{Security}
Das Programm hat keine speziellen Sicherheitsvorkehrungen. Es kommuniziert "uber ein IP-Netz mit den Programmen der anderen Spieler.

\section{Performance}
Mit den unter \ref{LeistungsAnforderungen} auf Seite \pageref{LeistungsAnforderungen} angegebenen Leistungsanforderungen soll ein angenehmes Spielverhalten gew"ahrleistet werden.
\section{Supportability}

\section{Free open source components}
Als Linux-Software-Developers liegt es Nahe, OpenSource Technologien zu verwenden.
(siehe dazu Projektplan, Kapitel Technologien)

\section{Developer Guidelines}

\subsection{Code Guidelines}

\begin{itemize}
  \item Methodennamen beginnen mit Kleinbuchstaben
  \item Klassennamen beginnen mit Grossbuchstaben
  \item Variabeln beginnen mit Kleinbuchstaben
  \item Vor und nach Operations-Zeichen wird ein Abstand gemacht.
  \item Nach zusammengeh"origen Code-Bl"ocken eine leere Zeile einf"ugen.
  \item Bei Klassen die "offnende geschweifte Klammer rechts neben dem Klassennamen und die schliessende auf eine eigene Zeile.
        F"ur alle anderen F"alle die "offnende und schliessende geschweifte Klammer auf eine eigene Zeile.
  \item Code innerhalb geschweifter Klammern wird einger"uckt.
  \item Falls Tabulatoren verwendet werden, in der Entwicklungsumgebung definieren, dass daf"ur Leerzeichen eingef"ugt werden.
  \item Default-Einr"uckung: Zwei Leerzeichen
  \item jede Kommentarzeile mit zwei slashes (//) beginnen.
\end{itemize}

\subsection{GUI Guidelines}
Folgt sp"ater in einem speraten Dokument.
