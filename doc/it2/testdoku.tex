\chapter{Abschlusstest}
\begin{tabular}{p{30mm}p{70mm}}
Durchgef"uhrt am: &  11.7.2002 \\
Tester          : &  U. Heimann  \\

\end{tabular}
\linebreak

\section{Normaler Spielablauf}

\subsection{Spiel starten als Server}
\begin{tabular}{|p{50mm}|p{70mm}|p{20mm}|}
\hline
\textbf{Aktion} & \textbf{Erwartetes Ergebnis} & \textbf{Ergebnis}  \\
\hline
Programm von der Konsole aus Starten.  &
Programm startet, GUI wird angezeigt.  &
OK \\
\hline
Spielername im Dialog 'Spieleinstellungen' setzen, mit '"Ubernehmen' best"atigen.  &
Name wird "ubernommen (nicht sichtbar), Dialog verschwindet.  &
OK  \\
\hline
Netzwerkserver starten, 'Netzwerk $\rightarrow$  Spielserver starten...'  &
Dialog 'Server starten' erscheint.  &
OK  \\
\hline
Mit 'Starten' den Server starten.  &
Server wird gestartet (nicht sichtbar), der Spieler wird am eigenen Server mit dem eingegebenen Namen
  angemeldet. Es erscheint ein Dialog der die Verbindung best"atigt. Der eingegebene Name erscheint
  zuoberst in der Playerliste.  &
OK \\
\hline
Warten bis sich ein weiterer Spieler angemeldet hat.  &
Der Name des anderen Spielers erscheint in der Playerliste.  &
OK  \\
\hline
Mit 'Spiel $\rightarrow$ Neues Spiel..' das Spiel starten.  &
Spielumgebung wird erstellt. Spielfelddaten werden "ubermittelt. Das Spielfeld wird angezeigt.
  Wenn s"amtliche Spieler bereit sind wird das Spiel freigegeben.  &
OK  \\
\hline
\end{tabular}

\subsection{Spiel starten als Client}
\begin{tabular}{|p{50mm}|p{70mm}|p{20mm}|}
\hline
\textbf{Aktion} & \textbf{Erwartetes Ergebnis} & \textbf{Ergebnis}  \\
\hline
Programm von der Konsole aus Starten.  &
Programm startet, GUI wird angezeigt.  &
OK \\
\hline
Spielername im Dialog 'Spieleinstellungen' setzen, mit '"Ubernehmen' best"atigen.  &
Name wird "ubernommen (nicht sichtbar), Dialog verschwindet.  &
OK  \\
\hline
An einem Server anmelden, 'Netzwerk $\rightarrow$ Spielserver anmelden...'  &
Dialog 'Server anmelden' erscheint.  &
OK  \\
\hline
IP Adresse des Servers eingeben und mit 'Anmelden' best"atigen.  &
Der Spieler wird am Server mit dem eingegebenen Namen angemeldet. Es erscheint ein Dialog der die
  Verbindung best"atigt. Der eingegebene Name erscheint in der Playerliste.  &
OK \\
\hline
Warten bis sich weitere Spieler angemeldet haben.  &
Der Name der anderen Spieler erscheint in der Playerliste.  &
OK  \\
\hline
Warten bis der Server das Spiel startet.  &
Spielumgebung wird erstellt. Spielfelddaten werden "ubermittelt. Das Spielfeld wird angezeigt.
  Wenn s"amtliche Spieler bereit sind wird das Spiel freigegeben.  &
OK  \\
\hline
\end{tabular}

\subsection{Spielen}

\begin{tabular}{|p{50mm}|p{70mm}|p{20mm}|}
\hline
\textbf{Aktion} & \textbf{Erwartetes Ergebnis} & \textbf{Ergebnis}  \\

\hline
Mit den Pfeiltasten wird die Spielfigur auf den freien Feldern bewegt.  &
Die eigene Spielfigur bewegt sich gem"ass den Anweisungen.  &
OK  \\
\hline
Mit der Leertaste wird eine Bombe gelegt.  &
Auf dem Spielfeld an der Position der Spielfigur wird eine Bombe angezeigt. &
OK  \\
\hline
Die Bombe explodiert nach einer bestimmten Zeit von selbst.  &
Auf dem Spielfeld wird der Feuerstrahl angezeigt. W"ande die vom Feuerstrahl getroffen werden, werden
  gel"oscht. Der Feuerstrahl verschwindet von selbst wieder.  &
OK  \\
\hline
Mehrere Bomben werden zeitlich versetzt nebeneinander gelegt.  &
Die Explosion der ersten Bombe l"ost die anderen Bomben ebenfalls aus.  &
OK  \\
\hline
Der andere Spieler bewegt seine Spielfigur und legt Bomben.  &
Die Spielfigur des Gegners bewegt sich auf dem Spielfeld gem"ass den Anweisungen des Gegners auf dem anderen
  Computer. Bomben werden angezeigt und explodieren. &
OK  \\
\hline
Eine Spielfigur wird von einem Feuerstrahl getroffen.  &
Die getroffene Spielfigur wird vom Spielfeld entfernt. Wenn nur noch ein Spieler auf dem Feld ist, erscheint
  eine Meldung mit dem Namen des Siegers. Das Spiel wird beendet. &
OK  \\
\hline
Der Server startet mit 'Spiel $\rightarrow$ Neues Spiel..' ein neues Spiel.  &
Ein neues Spielfeld wird angezeiget. Alle Spieler sind wieder dabei und k"onnen mitspielen.  &
OK  \\
\hline
\end{tabular}


\subsection{Spiel beenden}

\begin{tabular}{|p{50mm}|p{70mm}|p{20mm}|}
\hline
\textbf{Aktion} & \textbf{Erwartetes Ergebnis} & \textbf{Ergebnis}  \\
\hline
Ein Spieler (nicht der Server) schliesst seine Anwendung.  &
Der Spieler wird beim Server abgemeldet und das Programm wird geschlossen. Die verbliebenen Spieler
  erhalten eine Meldung, dass sich der Spieler abgemeldet hat. Der Name des Spielers verschwindet aus
  der Playerliste.  &
OK  \\
\hline
Der Spieler mit dem Server schliesst seine Anwendung.  &
Der Server meldet sich bei allen Clients ab und das Programm wird geschlossen. Die verbliebenen Spieler
  erhalten eine Meldung, dass sich der Server abgemeldet hat. S"amtliche Namen der Spieler verschwinden
  aus der Playerlist.  &
OK  \\
\hline
\end{tabular}


\section{Varianten Spielablauf}

\subsection{Spiel starten}

\begin{tabular}{|p{50mm}|p{70mm}|p{20mm}|}
\hline
\textbf{Aktion} & \textbf{Erwartetes Ergebnis} & \textbf{Ergebnis}  \\
\hline
Spiel vom Konqueror aus starten.  &
Spiel startet, GUI wird angezeigt.  &
FAILED   \\
 & & Das Spiel wird gestartet, die Grafiken f"ur das Spielfeld werden jedoch nicht gefunden. \\
\hline
Ein neuer Spieler versucht sich w"ahrend eines laufenden Spiels eine Verbindung aufzubauen.  &
Der Spieler kann die Verbindung aufbauen, kann aber keinen Einfluss auf das laufende Spiel nehmen.
  Beim n"achsten Spielstart spielt der Spieler auch mit.  &
OK  \\
\hline
Ein Spieler gibt vor dem Anmelden am Server keinen Spielernamen ein.  &
Er erscheint in der Playerliste als 'anonymous'.  &
OK  \\
\hline

\end{tabular}


\section{Fehlerf"alle}

\begin{tabular}{|p{50mm}|p{70mm}|p{20mm}|}
\hline
\textbf{Aktion} & \textbf{Erwartetes Ergebnis} & \textbf{Ergebnis}  \\
\hline
Der Spieler gibt beim Verbinden mit einem Server eine ung"ultige IP ein. &
Es wird eine Meldung ausgegeben, dass mit dem Server keine Verbindung aufgebaut werden konnte.  &
OK  \\
\hline
Mitten im Spiel schliesst ein Spieler (nicht der Server) seine Anwendung.  &
Die Figur des Spielers wird zerst"ort und verschwindet vom Spielfeld. Der Name des Spielers wird aus der
  Playerliste entfernt. Das Spiel l"auft weiter.  &
OK \\
\hline
Mitten im Spiel schliesst der Server seine Anwendung.  &
Das Spiel wird beendet. Die verbliebenen Spieler erhalten eine Meldung, und s"amtliche Namen werden aus
  der Playerliste gel"oscht.  &
OK \\
\hline
\end{tabular}


\chapter{Buglist}


\begin{tabular}{|p{5mm}|p{60mm}|p{65mm}|p{10mm}|}
\hline
\textbf{Nr.} & \textbf{Fehlerbild} & \textbf{M"ogliche Ursache} & \textbf{Status}  \\
\hline
1 &
Spielfiguren werden erst angezeigt wenn sie bewegt werden.  &
Positionen der Figuren werden beim Spielfeldaufbau nicht "ubermittelt.  &
FIXED  \\
\hline
2 &
Wenn ein Spieler die Verbindung abbricht wird der Name in der Playerliste nicht gel"oscht.  &
Im GUI wird auf das 'removePlayer' Ereigniss nicht reagiert.  &
FIXED  \\
\hline
3 &
Wenn das Programm beendet wird muss ca. 30 - 60 Sekunden gewartet werden bis wieder ein Server gestartet
  werden kann.  &
Eventuell wird der verwendete Socket vom System nicht unmittelbar freigegeben.  &
 \\
\hline
4 &
Wenn das Spiel vom Konqueror aus gestartet wird, wird das Spielfeld nicht angezeigt.
  Das Spiel funktioniert sonst normal.  &
Die Grafiken der W"ande, Mauern, etc. k"onnen nicht geladen werde. Eventuell falscher Arbeitspfad.  &
 \\
\hline


\end{tabular}


%\end{document}
