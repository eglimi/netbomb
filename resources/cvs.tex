\documentclass[a4paper]{article}
%Pakete auswaehlen
\usepackage{german}
\usepackage[latin1]{inputenc}
\usepackage{fancyhdr}
\usepackage{graphicx}
\usepackage{graphics}
\usepackage{hyperref}

\begin{document}
\title{CVS Zusammenfassung}
\author{Michael Egli}
\date{}
\maketitle

\begin{abstract}
Diese Zusammenfassung gilt, wenn ein Projekt auf \href{www.sourceforge.net}{www.sourceforge.net} verwaltet wird.
Konventionen f"ur diese Zusammenfassung:
\end{abstract}

\begin{itemize}
\item name steht f"ur den bei sourceforge angemeldeten Benutzernamen
\item projektname steht f"ur das bie sourceforge angemeldeten Projekt
\item directory steht f"ur das Verzeichnis, welches auf dem CVS Server ist.
\end{itemize}

\section{Voreinstellungen}
Wenn man sich bei sourceforge angemeldet hat, muss man sich als erstes mit ssh ins home directory
mit folgendem Befehl einloggen: \textit{ssh -l name projektname.sourceforge.net}

Im file ~/.bashrc folgendes eintragen:
\begin{itemize}
\item export CVS\_RSH=ssh
\item export CVSROOT=:ext:name@cvs.projektname.sourceforge.net:/cvsroot/projektname
\end{itemize}
Anschliessend die bash neu starten oder den PC neu booten, damit die "Anderungen "ubernommen werden.

Optional kann auch ein key erstellt werden. Das geschieht mit OpenSSH mit dem Befehl: \textit{ssh-keygen -t dsa}. Dieser Befehl
erzeugt einen ssh2 Key. (Es funktionieren auch ssh1 keys auf sourceforge).
DieserKey kann anschliessend auf der pers"onlichen account Seite raufgeladen werden. Das hat den Vorteil, dass man nicht immer
das Passwort eingeben muss.

\section{Projekt erstellen und importieren}
Zuerst wird ein Projekt erstellt (z.B. in Kdevelop). Da nur diese Verzeichnisse importiert werden sollen, die auch ge"andert werden,
das heisst ohne die generierten files, wechselt man in das Verzeichnis, das importiert werden soll.
Anschliessen kann mit dem Befehl \textit{cvs import directory vendor start} das Verzeichnis auf den CVS Server geladen werden. Dabei
ist directory der Name, wie er auf dem CVS Server heissen soll, \textit{nicht!} wie der Name des lokalen Verzeichnisses. <vendor>
ist irgendwas (spielt keine Rolle).

\section{Arbeiten mit CVS}
Um mit CVS arbeiten zu k"onnen muss zuerst das Verzeichnis vom CVS Server geladen werden. Hierzu dient der Befehl
\textit{cvs co directory}. Am besten legt man daf"ur ein neues Projektverzeichnis an, in dem fortan gearbeitet werden soll.
Verhindert werden muss auf jeden Fall, dass es das Verzeichnis lokal schon gibt, das heisst, wenn man das Projekt lokal erstellt hat
und es anschliessend auf den CVS Server geladen wird, muss dieses Verzeichnis zuerst gel"oscht und anschliessen
mit checkout wieder erstellt werden.

Zuerst sollte immer eine aktuelle Version (mit \textit{cvs up}) geholt werden. Dann k"onnen die files editiert werden. Wenn
das editieren abgeschlossen ist, kann wieder mit \textit{cvs up} gepr"uft werden welche Files sich ge"andert haben und anschliessend
kann mit \textit{cvs ci} die "Anderungen auf den CVS Server gespielt werden.

Fall neue Files oder Directories hinzugef"ugt werden sollen, muss dieses Directory zuerst erstellt werden. Anschliessend mit dem Befehl
\textit{cvs add file} und \textit{cvs ci file} wird das file oder directory auf den Server "ubertragen. Danach wieder mit \textit{cvs up}
das lokale Verzeichnis oder file auf den neuesten Stand bringen.

\section{Befehls"ubersicht}

$
\begin{array}{l l}
$cvs import directory vendor start$ & $Source Code ins CVS importieren$ \\
$cvs co <directory>$  & $Projekt auschecken (das wird nur einmal am Anfang gemacht)$ \\
$cvs up$ & $Update, "uberpr"ft "Anderungen die lokal gemacht wurden$ \\
$cvs ci$ & $Commit, "ubertr"agt die "Anderungen ins CVS$ \\
$cvs add file$ & $f"ugt dem CVS ein neues file oder directory hinzu$ \\
\end{array}
$

\end{document}
