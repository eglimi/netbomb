\documentclass[10pt]{report}
\usepackage{german}
\begin{document}

\chapter{Ausgangslage und Motivation}

Bomberman, ein unterhaltsames Spiel f"ur mehrere Spieler soll netztauglich und auf Linux portiert werden.
Bei herk"ommlichen Windows-Versionen f"ur mehrere Spieler gestaltet sich
die gleichzeitig an einer Tastatur erfolgende Bedienung als "ausserst unkonfortabel. Ebenso stehen oft nicht
gleich drei weitere Mitspieler vor Ort zur Verf"ugung, darum soll eine Netzwerk taugliche Version geschaffen werden.
Das Spiel hat keine komplizierten Spielregeln, ein paar wenige schnell erelenbare Funktionen.
Es soll mit einer ganzheitlich einfachen Handhabung realisiert werden. Damit man unverz"uglich in
den Mehrspieler-Spielgenuss eintauchen kann.
\\
Da wir unser eigener Auftraggeber sind, begr"undet sich unsere Motivation auch in der bew"altigung technologischer,
software-engineering orientierter und zwischenmenschlicher Aufgaben. Herausforderungen, die wir uns selber aufstellen.

\chapter{Features}
\section{Iteration 1}
\begin{itemize}{}{}
\item 1 Spielermodus
\item Spielfeld mit W"anden und Mauern
\item Spielfigur kann man bewegen
\item optional: Die Bewegungen des Spielers werden "ubers Netzwerk "ubertragen und auf einem anderen Rechner angezeigt
\end{itemize}

\section{Iteration 2}
\begin{itemize}{}{}
\item Es k"onnen Bomben gelegt werden, die explodieren
\item Spieler k"onnen sterben
\item Es k"onnen 4 Spieler zusammen spielen
\end{itemize}

\section{Optionen der Iteration 2}
\begin{itemize}
\item Es gibt Icons die einem spezielle F"ahigkeiten verleihen
\item Soundeffekte sind zu h"oren
\item Hintergrundmusik ist zu h"oren
\item es gibt einen Pausemodus
\item eine Highscore der besten Spieler ist verf"ugbar
\end{itemize}

\chapter{Herausforderungen}
\section{Implementation}
\begin{itemize}{}{}
\item Animierte Grafik und Sound auf Linux
\item Netzwerk Implementation
\item Ablauf der Synchronisation des Spieles zwischen verschiedenen Rechnern
\end{itemize}

\section{Technologie}
\begin{itemize}{}{}
\item Linux
\item Dokumentation mit LateX
\item Design Software Together
\end{itemize}


\section{anderer Art}
\subsection{Zielgerichtetes Arbeiten}
Durch die begrenzte, pro Woche zur Verf"ugung stehende Zeit(Ziel max. 4-8) und die vielseitige
 Aufgabenstellung, m"ussen wir zwangsl"aufig streng zielgerichtet arbeiten, um die Meilensteine
 erf"ullen zu k"onnen.

\subsection{Teamwork (2"<Teamitglieder)}
Ein Team mit mehr als zwei Personen erlaubt keine \textit{philosophischen}  Gruppendiskussionen mehr.
Es kann weder auf eine Traktandenliste f"ur Sitzungen, noch auf eine streng sachliche pro/contra
Argumentation w"ahrend einer Diskussion verzichtet werden.

\subsection{Kompromissbereitschaft}
Es muss von Perfektionismus zu notwendiger Zweckm"assigkeit hingearbeitet werden.

\subsection{Eigeninitiative}
Unser Team hat praktisch keine Erfahrungen aus gr"osseren Projekten mit gr"osseren Teams.
Gefragt ist Eigeninitiative zur Arbeit an eigener Teambereitschaft und Lernbereitschaft (nicht
nur f"ur Technologie sondern auch im Teambildungsprozess und Selbstorganisation).
\\
\textbf{Anpacken ist angesagt, aber nicht blind ohne Priorisierung und Ziel sondern
        mit Blick am Horziont und vereinten Kr"aften.
        }

\end{document}