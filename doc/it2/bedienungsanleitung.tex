\chapter{Installation und Bedienung}
	\section{Installation}	
		Die Installation erfolt wie bei den meisten Programmen unter Linux. Zuerst muss das tar Archiv (falls nicht schon entpackt)
		mit dem Befehl
		\begin{verbatim}
			tar -xvzf <name>.tgz
		\end{verbatim}
		entpackt werden. Anschliessend wechseln Sie in das neu erstellte Verzeichnis und kompilieren und installieren das Programm.
		Das geschieht mit den folgenden Befehlen
		\begin{verbatim}
			cd netbomber
			make
			make install
		\end{verbatim}

		Eine Ausf"uhrlichere Beschreibung finden sie in der mitgelieferten README Datei.

	\section{Bedienungsanleitung}
		Das Spiel \textsc{NetBomb} k"onnen Sie mit dem von der Konsole aus starten. Dazu wechseln Sie zuerst in das Verzeichnis,
		in das sie \textsc{NetBomb} installiert haben. Anschliessen geben Sie die Befehle
		\begin{verbatim}
			cd netbomber
			./netbomber
		\end{verbatim}
		ein. Damit wird das Programm gestartet. Es erscheint ein Fenster mit dem Spiel. 

		Als erstes k"onnen Sie unter dem Menu \textit{Optionen $\rightarrow$} Tastaturbelegung Ihren Namen eingeben.
		"Ubernehmen Sie Ihre Eingabe mit der Taste \textit{"Ubernehmen} und Beenden Sie den Dialog mit \textit{Abbrechen}.
		Anschliessend haben Sie zwei M"oglichkeiten. Sie k"onnen 
		\begin{enumerate}
			\item Einen Server starten, bei dem sich ihre Mitspieler anmelden k"onnen. Dazu gehen Sie folgendermassen vor
			\begin{enumerate}
				\item W"ahlen Sie  \textit{Netzwerk $\rightarrow$ Spielserver starten} und w"ahlen Sie dann die Anzahl Mitspieler.
							Anschliessen k"onnen Sie mit der Taste \textit{Starten} den Server starten.
				\item Es erscheint eine Meldung die Ihnen Mitteilt, ob sie erfolgreich eine Spielserver starten konnten. Best"atigen
							sie diesen Dialog. Danach erscheint in der Spielerliste (Liste rechts oben) ihr Name.
				\item Wenn Sie erfolgreich einen Server gestartet haben, m"ussen Sie warten, bis sich Ihre Mitspieler auf Ihren 
							Server eingeloggt haben. Sie sehen die Spieler, die Spielbereit sind in der Spielerliste.
				\item Sobald gen"ugend Spieler angemeldet sind, k"onnen sie mit \textit{Spiel $\rightarrow$ Neues Spiel} ein neues 
							Spiel starten.
				\item Have fun!
			\end{enumerate}
			\item Sich bei einem Server anmelden, um mitzuspielen. Dazu gehen Sie folgendermassen vor
			\begin{enumerate}
				\item Sie k"onnen sich mit \textit{Netzwerk $\rightarrow$ Spielserver anmelden} bei einem Server anmelden.
				\item	Es erscheint ein Fenster, in den Sie die IP - Adresse des Servers eingeben m"ussen. Tun Sie das und
							best"atigen Sie die Eingabe mit \textit{Anmelden} oder brechen Sie die Eingabe \textit{Abbrechen} ab.
				\item Sobald der Spieler, der den Server gestartet hat, beginnt das Spiel
				\item Have fun!
			\end{enumerate}
		\end{enumerate}

		
		\noindent
		Sie k"onnen das Spiel jederzeit mit \textit{Spiel $\rightarrow$ Beenden} schliessen. \\

		\noindent
		Viel Spass beim Spielen.
		Das \textsc{NetBomb} - Team
